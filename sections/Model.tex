\chapter{Model Description}

This section describes firstly the RNN structure and secondly the regularization.

\section{Recurrent Neural Network}

The recurrent neural network, RNN, is a nonlinear dynamical system mapping sequences to sequences. RNNs are typically structured with one input, one hidden and one output layer. Being parametrized by three weight matrices and two bias vectors \([W_{hi}, W_{hh}, W_{oh}, b_h, b_o]\) the state concatenation denoted \(\theta\) completely describes the RNN. Additionally an initial hidden state can be included but $h_0$ the initial hidden state is set to the zero vector.

The propagation through the network of an input sequence \(i_1, i_2, ... , i_T\) is performed in the following algorithm:

\begin{algorithmic}[1]
    \For{$t$ from $1$ to $T$}
        \State{$u_t \gets W_{hi}i_t + W_{hh}h_{t-1} + b_h$}
        \State{$h_t \gets a(u_t)$}
        \State{$o_t \gets W_{oh} + b_o $}
        \State{$z_t \gets s(o_t)$}
    \EndFor
\end{algorithmic}

Where $ a_h(*)$ and $a_o(*)$ are nonlinear activation functions for the hidden and output layer respectively. The cost function is usually defined as a sum of per-timestep cost:

\[C(z, y) = \sum_{t=1}^{T}C(z_t; y_t)\]

With the cost function a derivative for the RNNs can be computed by the backpropagation through time algorithm:

\begin{algorithmic}[1]
    \For{$t$ from $1$ to $T$}
        \State{$do_t \gets a_o'(o_t) dC(z_t; y_t)/dz_t $}
        \State{$db_o \gets db_o + do_t$}
        \State{$dW_{oh} \gets dW_{oh} + db_oh_t^\top $}
        \State{$dh_t \gets dh_t + W_{oh}^\top do_t$}
        \State{$du_t \gets a_h'(u_t)dh_t $}
        \State{$dW_{hi} \gets dW_{hi} + du_ti_t$}
        \State{$db_h \gets db_h + du_t $}
        \State{$dW_{hh} \gets dW_{hh} + du_th_{t-1}^\top $}
        \State{$ dh_{t-1} \gets W_{hh}^\top du_t$}
    \EndFor
    \State \Return{$d\theta = [dW_{hi}, dW_{hh}, dW_{oh}, db_h, db_o]$}
\end{algorithmic}

As mentioned there are difficulties in training RNNs. Included in this thesis is the Hessian-Free, HF, optimization, which is a second order method but combined with the R-method the above algorithm is essentially sufficient. However, as stated in the work, the approximation called the Gaus-Newton matrix, \(G\), of the Hessian is preferred for stability. Described in the algorithm below is the product \(Gv\) for some vector $v$:

\begin{algorithmic}[1]
    \For{$t$ from $1$ to $T$}
        \State{$do_t \gets a_o'(o_t) dC(z_t; y_t)/dz_t $}
        \State{$db_o \gets db_o + do_t$}
        \State{$dW_{oh} \gets dW_{oh} + db_oh_t^\top $}
        \State{$dh_t \gets dh_t + W_{oh}^\top do_t$}
        \State{$du_t \gets a_h'(u_t)dh_t $}
        \State{$dW_{hi} \gets dW_{hi} + du_ti_t$}
        \State{$db_h \gets db_h + du_t $}
        \State{$dW_{hh} \gets dW_{hh} + du_th_{t-1}^\top $}
        \State{$ dh_{t-1} \gets W_{hh}^\top du_t$}
    \EndFor
    \State \Return{$d\theta = [dW_{hi}, dW_{hh}, dW_{oh}, db_h, db_o]$}
\end{algorithmic}